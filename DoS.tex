\documentclass[twoside,colorback,accentcolor=tud4c,11pt]{tudreport}

\usepackage{ngerman}
\usepackage[utf8]{inputenc} 
\usepackage[T1]{fontenc}
\usepackage{cancel}
\usepackage{mathtools}
\usepackage{float}
\usepackage{hyperref}

\title{VERSUCH 4.12A: Design optischer Systeme}

\subtitle{
\begin{tabular}{p{4cm}ll} 
 Name & Ludwig Lind &   Jonas Fischer\\
 Matrikelnummer & •  & 2240758 \\
 E-mail& \textaccent{ludwig.lind@gmx.de} & \textaccent{jonas.fischer.42@gmail.com}\\
 \\Versuchsbetreuung & Jan Teske \\
 Durchführung& 08.05.2017 \\
 Abgabetermin& 29.05.2017
 \end{tabular}}
\institution{Institut für Angewandte Physik}
\sponsor{Wir erklären hiermit, dass die vorliegende Arbeit eigenständig, ohne fremde Hilfe und mit der angegebenen Literatur erarbeitet wurde. Alle Passagen aus Literatur und Internet sind als solche gekennzeichnet. Diese Arbeit liegt weder in gleicher noch ähnlicher Weise einer Prüfungskommission vor.\\\\ 
\begin{tabular}{lp{2em}lp{2em}l}
 \hspace{4cm}   && \hspace{4cm}  && \hspace{4cm}
 \\\cline{1-1}\cline{3-3}\cline{5-5}
    Darmstadt, den \today && Ludwig Lind && Jonas Fischer 
\end{tabular}  
 }   
\begin{document}

\maketitle 

\tableofcontents

\chapter{Ziel des Versuchs}
•
\chapter{Physikalische Grundlagen}
\section{Maxwell-Gleichungen und Wellengleichung}
\section{Physikalische Optik}
\section{Geometrische Optik}
Eikonal Theorie, Aberration
\section{Paraxiale Optik}
\section{Kenngrößen von optischen Systemen}
effektive Brennweite, Blendenzahl, Seidelkoeffizienten, Modulationsübertragungsfunktion,...
\chapter{Vorbereitende Aufgaben}
\section{Plankonvex- und Bikonvexlinse}
\section{Hubble-Teleskop}
\chapter{Versuchsdurchführung und Auswertung}
\section{Sammellinse}
\subsection{Paraxiale Linse}
\subsection{Reale Linse}
\subsection{Optimierung der Linse}
\subsection{Linsen-Doublett}
\section{Hubble-Teleskop}
\subsection{Paraxiales Teleskop}
\subsection{Reales Teleskop}
\subsection{Optimierung des Teleskops}
\subsection{Entwicklerversion}
\chapter{Fazit}	
•
\renewcommand{\bibname}{Literatur}
\begin{thebibliography}{0}
\bibitem {refname} •

\end{thebibliography}
\end{document}    
