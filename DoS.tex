\documentclass[twoside,colorback,accentcolor=tud4c,11pt]{tudreport}

\usepackage{ngerman}
\usepackage[utf8]{inputenc} 
\usepackage[T1]{fontenc}
\usepackage{cancel}
\usepackage{mathtools}
\usepackage{float}
\usepackage{hyperref}

\title{VERSUCH 4.12A: Design optischer Systeme}

\subtitle{
\begin{tabular}{p{4cm}ll} 
 Name & Ludwig Lind &   Jonas Fischer\\
 Matrikelnummer & •  & 2240758 \\
 E-mail& \textaccent{ludwig.lind@gmx.de} & \textaccent{jonas.fischer.42@gmail.com}\\
 \\Versuchsbetreuung & Jan Teske \\
 Durchführung& 08.05.2017 \\
 Abgabetermin& 29.05.2017
 \end{tabular}}
\institution{Institut für Angewandte Physik}
\sponsor{Wir erklären hiermit, dass die vorliegende Arbeit eigenständig, ohne fremde Hilfe und mit der angegebenen Literatur erarbeitet wurde. Alle Passagen aus Literatur und Internet sind als solche gekennzeichnet. Diese Arbeit liegt weder in gleicher noch ähnlicher Weise einer Prüfungskommission vor.\\\\ 
\begin{tabular}{lp{2em}lp{2em}l}
 \hspace{4cm}   && \hspace{4cm}  && \hspace{4cm}
 \\\cline{1-1}\cline{3-3}\cline{5-5}
    Darmstadt, den \today && Ludwig Lind && Jonas Fischer 
\end{tabular}  
 }   
\begin{document}

\maketitle 

\tableofcontents

\chapter{Ziel des Versuchs}
•
\chapter{Physikalische Grundlagen}
\section{Maxwell-Gleichungen und Wellengleichung}
\section{Physikalische Optik}
\section{Geometrische Optik}
\subsection{Eikonal Theorien lui}
\subsection{Aberration jonas}
\section{Paraxiale Optik}
ABCD
\section{Kenngrößen von optischen Systemen}
effektive Brennweite, Blendenzahl, Seidelkoeffizienten, Modulationsübertragungsfunktion,...
\chapter{Vorbereitende Aufgaben}
Im folgenden werden die zwei vorbereitenden Aufgaben behandelt
\section{Plankonvex- und Bikonvexlinse}
Um die Krümmungsradien der dünnen Linsen zu berechnen benutzen wir die Linsenmachergleichung
\begin{align}
\frac{1}{f}=(n-1)\left(\frac{1}{R_1}-\frac{1}{R_2}\right)
\end{align}
Gegeben waren n=1,5168, $ f=10 $ cm sowie $ R_2=\inf $ bzw. $ R_2=-R_1 $. Damit ergibt sich für die Plankonvexe Linse
\begin{align}
R_1&=5,168\text{ cm}\\
\text{bzw. }R_1&=10,336\text{ cm}
\end{align}
für die bikonvexe Linse.
\section{Hubble-Teleskop}
Zur Berechnung des Hubble-Teleskops bedienen wir uns der Matrizenoptik. Das Teleskop besteht aus zwei Propagationen und zwei Spiegeln
\begin{align}
T_1=\begin{pmatrix}
1&L\\
0&1
\end{pmatrix},\,
S_1=\begin{pmatrix}
1&o\\
-\frac{2}{R_1}&1
\end{pmatrix},\,
T_2=\begin{pmatrix}
1&6,4\text{ m}\\
0&1
\end{pmatrix},\,
S_2=\begin{pmatrix}
1&0\\
-\frac{2}{R_2}&1
\end{pmatrix}
\end{align} 
Die in \cite{anl} angegebenen Bedingungen ergeben dann
\begin{align}
T_2S_2T_1S_1\begin{pmatrix}
r_1\\
0
\end{pmatrix}
&=\begin{pmatrix}
0\\
\theta_1
\end{pmatrix}\\
T_1S_1\begin{pmatrix}
D_1/2\\
\theta_{\text{max}}
\end{pmatrix}
&=\begin{pmatrix}
D_2/2\\
\theta_2
\end{pmatrix}\\
T_2S_2T_1S_1\begin{pmatrix}
r_2\\
\theta_{\text{min}}
\end{pmatrix}
&=\begin{pmatrix}
r_{\text{min}}\\
\theta_3
\end{pmatrix}
\end{align}
wobei sich $ \theta_{\text{min}} $ durch das Rayleigh-Kriterium ergibt
\begin{align}
\theta_{\text{min}}=1,22\frac{\lambda}{D_1}=2,428\cdot 10^{-7}\text{ rad}
\end{align}
Durch lösen der Gleichungen erhält man
\begin{align}
R_1&=10,969\text{ m}\\
R_2&=-1,346\text{ m}\\
L&=4,876\text{ m}
\end{align}
\chapter{Versuchsdurchführung und Auswertung}
\section{Sammellinse}
\subsection{Paraxiale Linse}
\subsection{Reale Linse}
\subsection{Optimierung der Linse}
\subsection{Linsen-Doublett}
\section{Hubble-Teleskop}
\subsection{Paraxiales Teleskop}
\subsection{Reales Teleskop}
\subsection{Optimierung des Teleskops}
\subsection{Entwicklerversion}
\chapter{Fazit}	
•
\renewcommand{\bibname}{Literatur}
\begin{thebibliography}{0}
\bibitem {anl} Versuchsanleitung: Design optischer Systeme, Version vom 25. August 2015
\bibitem {skript} Vorlesungsskript Optik, WiSe 2016/17, Prof. T. Walther

\end{thebibliography}
\end{document}    
